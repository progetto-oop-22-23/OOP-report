\subsection{Michele Ceccacci}

\subsubsection{Logger} 
Problem: The system needs a composable logger, which has to track both state updates in devices and connections/disconnections
of devices. The need for composition stems from the fact that different use cases of the software might need just a subset
of the logging capabilities. The logger must also be dynamically composable for maximum flexibility.
Solution: used the Decorator design pattern, which allows to wrap dynamically objects at runtime and compose them.
Not using abstract classes also takes away the problems of inheritance. 
\begin{figure}[H]
\centering{}
\includegraphics[width=15cm]{uml/logger.png}
\caption{UML diagram of the logger}
\label{ceccacci:uml:logger}
\end{figure}

\subsubsection{Air conditioning and Heating}
Problem: 
Air conditioning and Heating devices have really similar behaviours. The difference between the two, is that an air conditioning
device decreases the environment's temperature, whereas the heating device raises it.
Solution: Since the only function that actually changes between the two implementations is the one responsible for updating the
environment's temperature, the template method pattern was used. 
The method updateTick, responsible to update the environment and electrical consumption when called by the controller,
is a template method, and calls the abstract method updateTemperature.

\begin{figure}[H]
\centering{}
\includegraphics[width=15cm]{uml/abstractTemperaturechanger.png}
\caption{TemperatureChangers package uml diagram}
\label{ceccacci:uml:temperaturechangers}
\end{figure}

\subsubsection{Passing state updates to the view}
Problem: devices have really different states, and they all need to passed down to the view.
I took inspiration from REST applications, and thought about implementing a  serialization/deserialization protocol.
Since the view and the model don't need to communicate over a network, in the end i decided to just implement a DeviceState interface, 
which all device states implement. This also respects the single responsibilty principle, since every device's state can be updated independently.
Then i defined a single command for state updates, so that the view could create a component for the specific device in case
the device wasn't already present. To handle device removals, i just created a single removeDevice Method in the DeviceViewer interface, 
based on the deviceId, since it identifies the device univocally. \newline 
Another problem related to this, is the fact that when a view sends an update to the controller, all the other views need to be notified
too. This was solved by adding a ViewManager to the controller, in charge of passing down updates to all the views. 
ViewManager extends the View interface, so we can be sure that all the valid view updates are also valid DeviceManager updates.
The other only method ViewManager defines is addView, allowing to add other views dynamically.

\begin{figure}[H]
\centering{}
\includegraphics[width=15cm]{uml/viewmanager.png}
\caption{ViewManager uml diagram}
\label{ceccacci:uml:view}
\end{figure}
