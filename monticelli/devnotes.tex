\subsection {Alessandro Monticelli}
\begin{itemize}
    \item \textbf{Generics} Use of generics was fundamental for keeping a clean clode and reach high extensibility.
                            \url {https://github.com/progetto-oop-22-23/OOP22-HOMER/blob/8779c6742ac88985e781856c26cf6815686ea9c7/src/main/java/homer/model/lights/Light.java#L21}
    \item \textbf{Optional}
                            \url {https://github.com/progetto-oop-22-23/OOP22-HOMER/blob/315b0f02d5e7a9dc0812e59ee195f5052ce3ddb1/src/main/java/homer/controller/electricalmeter/ElectricalMeterImpl.java#L37}
    \item \textbf{Stream} The use of streams allowed me to keep my code clean and readable, I made large use of this Java feature, 
                            which I think is among the most interesting.
                            \url {https://github.com/progetto-oop-22-23/OOP22-HOMER/blob/8779c6742ac88985e781856c26cf6815686ea9c7/src/main/java/homer/controller/electricalmeter/ElectricalMeterImpl.java#L109}
    \item \textbf{Lambda}
                            \url {https://github.com/progetto-oop-22-23/OOP22-HOMER/blob/315b0f02d5e7a9dc0812e59ee195f5052ce3ddb1/src/main/java/homer/view/javafx/sensorsview/ElectricalMeterViewManager.java#L100}
    \item \textbf{JavaFX}
        \begin{itemize}
            \item \textbf{Scene Builder} It proved to be a fundamental tool for building the view. I quickly learned to use it, 
                                            and it allowed me to create a consistent UI without too much hassle.
            \item \textbf{FXML} I really like the idea of creating the UI through a structured markup language. I think it's functional and clean.
                                I also appreciate the way the view is managed through a controller class.
                                \url {https://github.com/progetto-oop-22-23/OOP22-HOMER/blob/main/src/main/resources/homer/view/javafx/sensorsview/ElectricalMeterView.fxml}
        \end{itemize}
\end{itemize}