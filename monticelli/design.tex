\subsection{Alessandro Monticelli}
\subsubsection*{Electrical Meter}
When we first ideated HOMER, we agreed to build an Electrical Meter allows the user check the electrical consumption of various devices during the simulation.

Its main role is to retreive the state of the Outlets and calculate the istantaneous power absorption and the overall electrical consumption.
It also implements an automatic safety check on the consumption as it will cut the power to the most consuming Outlet when the global power absorption reaches the limit of 4kW.
The Electrical Meter acts as part of the Controller in the MVC pattern. 

I implemented the ElectricalMeter interface in the ElectricalMeterImpl class, which updates its state in the updateTick method, where it retreives the state of the Outlets, 
performs the safety check, and the notifies the View through the view manager, that keeps the references to the elements of the View for the FXML loader, 
and implements the methods for retreiving User input and updating the View.

\begin{figure}[H]
    \centering{}
    \includegraphics{uml/ElectricalMeterSimple.png}
    \caption{ElectricalMeter simple diagram}
    \label{monticelli:uml:simpleMeter}
\end{figure}

\begin{figure}[H]
    \centering{}
    \includegraphics{uml/ElectricalMeterDetailed.png}
    \caption{ElectricalMeterImpl detailed diagram}
    \label{monticelli:uml:detailedMeter}
\end{figure}

\subsubsection{Outlets}
As mentioned earlier, one of the key features of HOMER is the ability to control electrical equipment. \newline
To accomplish this, I created the Outlet class, which is modeled as an AdjustableDevice. \newline
The Outlet's state is represented by the OutletState class, which tracks the consumed power over time in Watt-hours (Wh) 
from any device that is plugged into the Outlet. Additionally, OutletState stores the minimum and maximum power consumption 
values for the device. \newline

To maintain a realistic simulation, the Outlet class does not have knowledge of whether a device is plugged in or not. 
Instead, I designed the Outlet to let devices set their own power consumption and send that information to the Outlet when needed. 
To facilitate standardization of Outlets, I implemented the Factory pattern, which provides two standard options for creating Outlets:
C-type Outlets (also known as Schuko), with a maximum power absorption of 3.5 kW, and L-type Outlets (the standard Italian outlet),
with a maximum power absorption of 2 kW.
\begin{figure}[H]
    \centering{}
    \includegraphics{uml/Outlet.png}
    \caption{Outlet simple diagram}
    \label{monticelli:uml:outlet}
\end{figure}
\subsubsection*{Powered Devices}
In order to model power-consuming devices, I created the PoweredDevice and PoweredDeviceInfo interfaces.\newline
The PoweredDevice interface includes methods for controlling the device's power consumption, retrieving the corresponding PoweredDeviceInfo object, 
and setting a new outlet for the device. Meanwhile, the PoweredDeviceInfo interface provides methods for retrieving the outlet that the device is plugged into, 
as well as the minimum and maximum power consumption values for the device.

\begin{figure}[H]
    \centering{}
    \includegraphics{uml/PoweredDevice.png}
    \caption{PoweredDevice and PoweredInfo diagrams}
    \label{monticelli:uml:poweredDevice}
\end{figure}
\subsubsection*{Lights}
The Light class represent a very basic implementation of a ToggleableDevice and a PoweredDevice, its state is held in the LightState class, which has methods to retreive the state of the Light (ON/OFF).

\begin{figure}[H]
    \centering{}
    \includegraphics{uml/Light.png}
    \caption{ElectricalMeter simple diagram}
    \label{monticelli:uml:light}
\end{figure}
