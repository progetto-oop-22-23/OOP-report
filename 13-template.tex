\documentclass[a4paper,12pt]{report}

\usepackage{alltt, fancyvrb, url}
\usepackage{graphicx}
\usepackage[utf8]{inputenc}
\usepackage{float}
\usepackage{hyperref}

% Questo commentalo se vuoi scrivere in inglese.
% \usepackage[italian]{babel}

% \usepackage[italian]{cleveref}
\usepackage{cleveref}

\title{Homer}

\author{Ceccacci Michele, Magnani Simone, Monticelli Alessandro}
\date{\today}


\begin{document}

\maketitle

\begin{abstract}
\end{abstract}

\tableofcontents

\chapter{Analysis}
\section{Requirements}

HOMER (HOMe EmulatoR) is an emulated domotic controller connected to lights, windows, and other domestic facilities.
The smart environment is mananged through a dashboard which allows the user to control devices, monitor sensors and electrical consumptions.
Each device should support updates in variable time units.
Some devices should act in a somewhat random way (but still predictable), while others should be fully deterministic.

\subsubsection{Functional Requirements}

\begin{itemize}
	\item The software should allow creating and controlling lots of different devices, such as:
	\begin{itemize}
		\item Lights
		\item Electrical outlets
		\item Doors, Locks and windows
		\item Temperature changer devices
		\item Air quality sensors
	\end{itemize}
	\item Devices can be added/removed at runtime.
	\item A logger is needed in order to allow users (and possibly machines) to keep track of state changes.
	\item The software should have graph views for temperature and air quality state.
	\item The users should be able to see electrical consumption.
\end{itemize}

\subsubsection{Non-Functional Requirements}

\begin{itemize}
	\item Should performant enough to run on a desktop/laptop.
	\item User interface should be intuitive and user-friendly
	\item The project should be portable, and work on Windows, MacOS and Linux devices.
\end{itemize}

\section{Analysis and domain model}

The application will emulate a domotic environment, some smart devices and facilities monitoring.
Smart devices are common domestic devices as lights, electrical outlets, windows, doors and various sensors 
for monitoring indoor temperature and air quality along with electrical absorption.
The user will interact with the devices via the controller through a dashboard.

The main challenge will be modelling the communication between the controller 
and the different devices, which can have very different states.

\begin{figure}[H]
\centering{}
\includegraphics{img/analysis.pdf}
\caption{Schema UML dell'analisi del problema, con rappresentate le entità principali ed i rapporti fra loro}
\label{img:analysis}
\end{figure}

\chapter{Design}
\section{Architecture}

HOMER is built on the MVC (model-view-control) architecture. 
In order to start JavaFX, JFXapplication will be our application's entry point.
JFXApplication will instance the controller, which is responsible for getting inputs from the views,
and updating the model consequently. The controller also sends back state updates to the various views.
The model part is composed by the devices and electrical outlets.


\section{Detailed Design}
Problem: Sending updates from the view to the controller. 
We need a common way for all the UI components to send the same kind of updates to the controller without repeating code.
We also want the controller to have more control over what is actually execute, and when it is executed. So the controller acts as a receiver.
All the commands implement an execute method, that is called inside the controller.
Solution: Used the command pattern

\subsection{Michele Ceccacci}

\textbf{Logger} \newline
Problem: The system needs a composable logger, which has to track both state updates in devices and connections/disconnections
of devices. The need for composition stems from the fact that different use cases of the software might need just a subset
of the logging capabilities. The logger must also be dynamically composable for maximum flexibility.
Solution: used the Decorator design pattern, which allows to wrap dynamically objects at runtime and compose them.
Not using abstract classes also takes away the problems of inheritance. \newline
\includegraphics[width=16cm]{uml/logger.png}

\textbf{Air conditioning and Heating}\newline
Problem: 
Air conditioning and Heating devices have really similar behaviours. The difference between the two, is that an air conditioning
device decreases the environment's temperature, whereas the heating device raises it.
Solution: Since the only function that actually changes between the two implementations is the one responsible for updating the
environment's temperature, the template method pattern was used. 
The method updateTick, responsible to update the environment and electrical consumption when called by the controller,
is a template method, and calls the abstract method updateTemperature. \newline

\includegraphics[width=16cm]{uml/abstractTemperaturechanger.png}

\subsection{Simone Magnani}

\subsubsection{Simulation}

The simulation is a \texttt{discrete-time simulation}, where each tick represents an amount of time that has passed.
This allows to control the \texttt{delta time}, therefore allowing to speed up/down the simulation.

Any component that requires to update its state in steps and know how much time
has passed will implement the interface \texttt{DiscreteObject}.

\begin{figure}[H]
\centering{}
% \includegraphics[width=\textwidth,height=\textheight,keepaspectratio]{magnani/uml/discreteobject.png}
\includegraphics[keepaspectratio]{magnani/uml/discreteobject.png}
\caption{UML diagram of the DiscreteObject interface}
\label{magnani:uml:discreteobject}
\end{figure}

\paragraph{Problem} Since the simulation can be sped up, we have to track the time in the simulation environment
\paragraph{Solution} Creation of the concept of \texttt{Clock} that stores the time and is updated at each tick.

\begin{figure}[H]
\centering{}
% \includegraphics[width=\textwidth,height=\textheight,keepaspectratio]{magnani/uml/clock.png}
\includegraphics[keepaspectratio]{magnani/uml/clock.png}
\caption{UML diagram of the clock}
\label{magnani:uml:clock}
\end{figure}

\subsubsection{Control of the simulation}

The game loop is handled in the simulation manager.

The simulation manager logic is split between the \texttt{SimManager} and \texttt{SimManagerViewObserver}.

Any \texttt{DiscreteObject} can ask to be updated by the game loop with the
\texttt{Observer} pattern,via the \texttt{SimManager} method \texttt{addObserver}.

The simulation manager interfaces with its view counterpart to allow pause/resume
and time rate control via the \texttt{SimManagerViewObserver} using the \texttt{Observer} pattern.

\begin{figure}[H]
\centering{}
\includegraphics[width=\textwidth,height=\textheight,keepaspectratio]{magnani/uml/simmanager.png}
\caption{UML diagram of the SimulationManager}
\label{magnani:uml:simmanager}
\end{figure}

\subsubsection{Devices}

Devices will be detailed separarely between read-only, toggleable and adjustable ones.

\subsubsection{Thermometer}

The thermometer is modelled as an extension of \texttt{Device} since it's not meant to be controllable
but only to return a \texttt{DeviceState} to the Controller,
and \texttt{DiscreteObject}, to define how should the temperature be sensed (in
a simple case, it could be updated directly, or in another case imperfections such as lag can be modelled).

% The DiscreteObject could also have been directly implemented instead of being extended by Thermometer,
% but I wanted to put 

\begin{figure}[H]
\centering{}
\includegraphics[width=\textwidth,height=\textheight,keepaspectratio]{magnani/uml/thermometer.png}
\caption{UML diagram of the thermometer}
\label{magnani:uml:thermometer}
\end{figure}

\paragraph{Problem} There are multiple devices that need to interact with a common temperature
\paragraph{Solution} Creation of the concept of \texttt{Environment} which represents the
physical simulated environment with its temperature parameter.
In the future, this would also allow to simulate different environments, eg. different rooms or completely
separate buildings.

\begin{figure}[H]
\centering{}
\includegraphics[keepaspectratio]{magnani/uml/environment.png}
\caption{UML diagram of the environment}
\label{magnani:uml:environment}
\end{figure}

\subsubsection{Lock}

The Lock has been modelled as a \texttt{ToggleableDevice}.
Its logic is really simple, the state can vary between locked and unlocked.
The lock is represented in figure \Cref{magnani:uml:lock}

\begin{figure}[H]
\centering{}
\includegraphics[width=10cm,height=\textheight,keepaspectratio]{magnani/uml/lock.png}
\caption{UML diagram of the lock}
\label{magnani:uml:lock}
\end{figure}

\subsubsection{Window, Door, Blinds}

The devices of type \texttt{Window, Door and Blinds} are modelled as \texttt{AdjustableDevice}s.

\paragraph{Problem} Each of these devices can be considered to be controllable between a range of values. Also, if those devices
were meant to be remotely controllable, there should be something that moves them.
\paragraph{Solution} Create the concept of \texttt{Actuator}. This also allows someone to decide however they want to model
the movement mechanism (as simplistically or realistically as preferred) $=>$ \texttt{Strategy} pattern.
It also allows to compartmentalize away the logic of the movement, adhering to the single responsibility principle.

\paragraph{Problem} Min, max values for ranges (eg. actuator movement range) require to be managed
in several parts. It is also necessary to make sure that the order of the values is correct.
This would lead to a lot of duplicated code.
\paragraph{Solution} Create a \texttt{Bounds} class encapsulating the concept of boundaries,
which also allows to check the correct order of the bounds internally.

\begin{figure}[H]
\centering{}
\includegraphics[width=\textwidth,height=12cm,keepaspectratio]{magnani/uml/actuator.png}
\caption{UML diagram of the actuator}
\label{magnani:uml:actuator}
\end{figure}

\paragraph{Problem} Reuse of code for the different devices.
\paragraph{Solution} Create the concept of \texttt{ActuatedDevice}. Then, create an \texttt{AbstractActuatedDevice}
which wraps an \texttt{Actuator} (\texttt{Decorator} pattern).
It is then possible to create several different implementations and choose which actuator implementation to use,
without having to create other device implementations that would be practically identical.

% Problem: Actuator in MechanizedWindow (we might want to change it eg. use a more realistic one)
% Solution: pass the actuator in the constructor, this allows to use different actuators without having
% to create other Window implementations that would only be different in terms of which actuator is being used.

\begin{figure}[H]
\centering{}
\includegraphics[width=\textwidth,height=\textheight,keepaspectratio]{magnani/uml/actuateddevice.png}
\caption{UML diagram of the ActuatedDevice and how the devices derive from it}
\label{magnani:uml:actuateddevice}
\end{figure}

\subsubsection{Temperature Scheduler}

% Problem: scheduler actions
% Solution: Command pattern

% Problem: generalize commands for both heating and cooling
% Solution: Function

% Problem: adding of time (overlapse between begin and end, and at 24h/00h of next day)

\subsubsection{Graphs}

As with the scheduler, the implementation of the graphs has been divided between model, controller, and view components.

\subsubsection{Graphs Model}

The model has the simple task of storing the logged data, and providing the dataset upon request.

\begin{figure}[H]
\centering{}
\includegraphics[width=\textwidth,height=\textheight,keepaspectratio]{magnani/uml/graph-model.png}
\caption{UML diagram of the graph model}
\label{magnani:uml:graph-model}
\end{figure}

\subsubsection{Graphs Controller}

The controller has the task of supplying the data to log to the model, and to refresh the view.
It has been generalized with the use of a \texttt{Supplier} in an abstract implementation \texttt{AbstractLogger}. \newline
The supplier decides how to retrieve the data to log, effectively becoming an abstract method, and the updateTick (where the data is logged)
becomes a template method.

\begin{figure}[H]
\centering{}
\includegraphics[width=\textwidth,height=\textheight,keepaspectratio]{magnani/uml/graph-controller.png}
\caption{UML diagram of the graph controller}
\label{magnani:uml:graph-controller}
\end{figure}

\paragraph{Problem} We have to choose which thermometer to log.
\paragraph{Solution} Use the first/any thermometer found.
This solution would not work with multiple thermometers in different environments.
In that case it would be better to choose a particular thermometer to log, and/or to have multiple logs for all the different thermometers.

\paragraph{Problem} We have to choose how to log the data.
\paragraph{Solution} I chose to log the data at each (simulation) hour for design simplicity.
Could also have logged each tick, but it would have led to some pretty confusing graphs if the time rate were to be modified
due to the time axis not being linear.
In the future, the time between each log could be configured.
It could also be possible to log more frequently but choose to display at a larger interval of time.

\subsubsection{Graphs View}

The graph view has the task of displaying the historical data, sent from the controller, to the user.

\begin{figure}[H]
\centering{}
\includegraphics[width=\textwidth,height=\textheight,keepaspectratio]{magnani/uml/graph-view.png}
\caption{UML diagram of the graph view}
\label{magnani:uml:graph-view}
\end{figure}

\paragraph{Problem} How to handle the display of the new data.
\paragraph{Solution} I chose to refresh all the data each update for design simplicity.
Another way would have been to send only the new data from the graph controller to the view,
this would have required the view to decide whether to discard the oldest data record,
and due to time constraints, I just decided to go with the simplest way.
It also allowed me to better separate the logic and view, in fact it's the controller
deciding what to display.



\chapter{Development}
\section{Automated testing}
We used unit testing especially on the model to prevent regressions, and we tested observable behaviour only.
Our unit test suite uses JUnit.
We decided not to test UI components, since our UI is not the main foucs,
and other areas of the model would benefit more from more granular testing. 
Tested classes:
\begin{itemize}
	\item Temperature
	\item DurationConverter
	\item ElectricalMeter
	\item Lights
	\item Outlets
	\item temperaturechangers
	\item Logger
\end{itemize}

\subsection{Michele Ceccacci}
\begin{itemize}
	\item TemperatureTest
	\item AirqualityStateTest
	\item AbstractTemperatureChangerTest
	\item AirConditioningTest
	\item HeatingTest
	\item LoggerImplTest
\end{itemize}

\subsection{Simone Magnani}
\begin{itemize}
	\item BoundsTest
	\item LimitTest
	\item ClockImplTest
	\item SimpleActuatorTest
	\item AbstractActuatedDeviceTest
	\item SimpleLockTest
	\item TemperatureSchedulerTest
	\item MechanizedWindowTest
\end{itemize}


\section{Workflow}
The first step in our process was domain analysis, and then the insights gained were applied by modeling 
the project's main interfaces by using UML.
We used Git as our DCVS. We used gitflow and feature branching, with dev as our development branch,
and main as our stable branch for releases. The workflow is based on pull requests and forks, allowing for asynchronous feedback when needed.
Group meeting were only needed to discuss refactors. 

\subsection{Michele Ceccacci}
\begin{itemize}
	\item Add/remove devices interface.
	\item Actuated devices view
	\item Airconditioning/Heating view and model implementations
	\item Air quality view and model implmentation
	\item Application Logger 
\end{itemize}
\section{Development notes}

\subsection{Simone Magnani}
\begin{itemize}
    \item Utility classes
    \begin{itemize}
        \item \texttt{homer.common.bounds.*} for the abstraction of boundary values
        \item \texttt{homer.common.history.*} for the concept of data associated to a time
        \item \texttt{homer.common.limit.*} for a common and reusable way to limit values
        \item \texttt{homer.common.time.Clock*} for the concept of tracking custom time
    \end{itemize}
    \item For the simulation core
    \begin{itemize}
        \item \texttt{homer.core.*}
        \item \texttt{homer.view.sim.*}
    \end{itemize}
    \item For the implementation of the temperature scheduler
    \begin{itemize}
        \item \texttt{homer.controller.scheduler.*}
        \item \texttt{homer.model.scheduler.*}
        \item \texttt{homer.view.scheduler.*}
    \end{itemize}
    \item For the implementation of the graphs
    \begin{itemize}
        \item \texttt{homer.controller.history.*}
        \item \texttt{homer.model.history.*}
        \item \texttt{homer.view.graph.*} 
    \end{itemize}
    \item Devices
    \begin{itemize}
        \item \texttt{homer.model.lock.*}
        \item \texttt{homer.model.thermometer.*}
        \item \texttt{homer.model.environment.*} for the concept of physical environment
        \item \texttt{homer.model.actuator.*} for the concept of actuator and the abstraction of actuated devices
        \item \texttt{homer.model.blinds.*}
        \item \texttt{homer.model.door.*}
        \item \texttt{homer.model.window.*}
    \end{itemize}
\end{itemize}

I have contributed to the establishing of the architecture eg. \texttt{Device} and \texttt{Controller} interfaces.

When I had some ideas that could be used by someone else eg. \texttt{Environment, Bounds, Limit},
I would open a pull request, which allowed everyone to be aware of it, and who could chime in and provide their feedback.

I have also contributed in minor parts to the entrypoint and \texttt{Controller} implementation, for the integration of my code,
and \texttt{JFXDeviceViewer}.


\section{Development Notes}

\subsection{Michele Ceccacci}
\begin{itemize}
	\item \textbf{Optional}s were used when the value could either be present or not
	\item \textbf{Streams} were used to access and modify data, especially data that used optional types
	\item \textbf{Records} to make data immutability more explicit, and to avoid reimplementing constructors/HashCode methods
	\item \textbf{Lambda} functions and functional interfaces were used to make code more readable and often complemented streams 
	\item \textbf{JavaFX} was used to develop the main view.
\end{itemize}

\subsection{Simone Magnani}

\begin{itemize}
    \item \textbf{Generics and bounded generics} used in several parts of my code
    \begin{itemize}
        \item \url{https://github.com/progetto-oop-22-23/OOP22-HOMER/blob/main/src/main/java/homer/common/limit/Limit.java#L23}
    \end{itemize}
    \item \textbf{Lambda expressions}
    \begin{itemize}
        \item \url{https://github.com/progetto-oop-22-23/OOP22-HOMER/blob/main/src/main/java/homer/controller/scheduler/TemperatureCommandsImpl.java#L12}
    \end{itemize}
    \item \textbf{Functional interfaces} such as \texttt{Function}, \texttt{Supplier}, \texttt{BiConsumer}
    \begin{itemize}
        \item \url{https://github.com/progetto-oop-22-23/OOP22-HOMER/blob/main/src/main/java/homer/controller/scheduler/TemperatureCommand.java#L45}
    \end{itemize}
    \item \textbf{Stream}
    \begin{itemize}
        \item \url{https://github.com/progetto-oop-22-23/OOP22-HOMER/blob/main/src/main/java/homer/model/scheduler/TimeSchedulerImpl.java#L57}
    \end{itemize}
    \item \textbf{Optional}
    \begin{itemize}
        \item \url{https://github.com/progetto-oop-22-23/OOP22-HOMER/blob/main/src/main/java/homer/controller/scheduler/TemperatureSchedulerController.java#L81}
    \end{itemize}
    \item \textbf{Record} for the implementation of \texttt{HistoryData}
    \begin{itemize}
        \item \url{https://github.com/progetto-oop-22-23/OOP22-HOMER/blob/main/src/main/java/homer/common/history/HistoryData.java#L12}
    \end{itemize}
    \item \textbf{JavaFX} for the implementation of the sim, scheduler and graph views
    \begin{itemize}
        \item \url{https://github.com/progetto-oop-22-23/OOP22-HOMER/blob/main/src/main/java/homer/view/sim/SimManagerViewFxImpl.java#L20}
    \end{itemize}
    \item \textbf{ControlsFX} for the range slider
    \begin{itemize}
        \item \url{https://github.com/progetto-oop-22-23/OOP22-HOMER/blob/main/src/main/java/homer/view/scheduler/AddTemperatureScheduleViewFx.java#L28}
    \end{itemize}
\end{itemize}

To allow the air quality graphs FlowPane (placed inside of a ScrollPane) to automatically resize these two lines were used \url{https://stackoverflow.com/a/36264110}.


\chapter{Final comments}

\section{Self evaluation and final comments}

\subsection{Michele Ceccacci}
This project definitely improved my teamworking and software architecture skills. 
This was my first time participating in a greenfield project this big, and my first time 
taking architectural decisions that would carry over the whole software's lifecycle. 
In my previous work experience and open source contriubtions, i was more focused on  adding new features 
to an existing project or fixing tickets, rather than actual softare architecture. 
I was familiar proficient at git, but i still made some occasional mistakes.
I think that working in a team without a more senior figure acting as the  lead definitely 
made me more  self reliant, and allowed me to dig deeper into problems.

\subsection{Simone Magnani}

Although this was not my first project of such dimensions to be designed from
scratch, it was the first one to be made in a team, and the first one where I
had the awareness that certain design choices had a name and are commonly used.
Altough the minimum requirements were met, I expected to reach further.
I now feel to have grasped how git works and am comfortable in using it in collaborative projects.
I will definitely carry over the knowledge I have acquired in architecture design and design patterns
onto my future projects.


\appendix
\chapter{User guide}

In the main tab, there are 3  tabs: devices, scheduler and graphs.
In the devices tab, there is an add window button. After selecting a device,
it will be created (and will be inserted at the bottom). Each device has a remove device button
used to disconnect it, and may have some kind of utility to set its state. \newline
At the bottom there are the scheduler buttons, which allow the user to set scheduler speed, pause it and resume it.
In the scheduler view, the user can select time slices for heating. \newline % TODO this needs to be explained better.
In the graph view, temperature and air quality stats are displayed (provided the user already connected a related sensor).

%TODO: describe outlet view.

\chapter{Lab Assignments}

\section{simone.magnani7@studio.unibo.it}

\begin{itemize}
 \item Laboratorio 07: \url{https://virtuale.unibo.it/mod/forum/discuss.php?d=117044#p173103}
 \item Laboratorio 09: \url{https://virtuale.unibo.it/mod/forum/discuss.php?d=118995#p175305}
 \item Laboratorio 10: \url{https://virtuale.unibo.it/mod/forum/discuss.php?d=119938#p176561}
 \item Laboratorio 11: \url{https://virtuale.unibo.it/mod/forum/discuss.php?d=121130#p177406}
\end{itemize}


\bibliographystyle{alpha}
\bibliography{13-template}

\end{document}
